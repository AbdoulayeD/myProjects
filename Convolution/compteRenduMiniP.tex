ss[a4paper, 10pt]{article}
\usepackage[francais]{babel}
\usepackage{amsmath}
\usepackage{graphicx}
\title{\textbf{Mini Projet GLCS: Convolution Parallele pour le Traitement d'image}}

\author{\\
R\'ealis\'e par: Abdoulaye Diop\\
}
\date{}

\begin{document}

\maketitle
\mbox{}
\cleardoublepage
\renewcommand{\contentsname}{Sommaire}
\tableofcontents

\cleardoublepage
\section{Introdution}
L'objectif de ce projet  est de r\'ealiser from scratch un programme C++ de convolution paralléle
op\'erant sur des images de format PIXMAP(pgm et ppm).Cette opération consiste \'a appliquer
des op\'erations de filtrage sur un image, en fonction du type de kernel de convolution (type de filtre) choisi.
il s'agira ainsi  d'\`elaborer un code qui respectera le cahier de charge suivant:
\begin{itemize}
\item Gestion du format portable pixmap
\item Parallélisme multi-niveaux et réutilisable
\item Taille de kernel variable
\item Choix des filtres (filter blur, motion blur, sharpen...)
\item Choix des types de flottants
\end{itemize}

Ce rapport est un explicatifs des différents choix d'implémentation réaliser pour statifaire au cahier des charges
pr\'ecedent. Il présentera aussi des benchmark de temps d'execution et seepdup entre les versions paralléles et
sequentielle.

\section{Convolution}
En traitement d'image l'operation de convolution est une opération d'application de filtre sur image de depart.
Elle est basé sur l'utilisation de noyau de convolution (kernel), qui sont sous la forme d'une 
matrice de donné.
On applique ainsi cette opération de filtrage en :
\begin{itemize}
\item Recuperant de l'image, un block de pixel sous formes matricielles de taille kernel.size()
\item En appliquant des multiplications des elements du kernel au element du blocks.
\item En enregistrant le résultat obtenue dans la cellule centrale du block.
\item on itére ainsi avec un pas de 1 {\it pixel} pour réaliser l'opération sur toute l'image
\end{itemize}

\subsection{Remarque}
A un resultat partiel de la convolution, on peut ajouter des paramettres tels qu'un diviseur({\it factor}, et un terme de 
décalage{\it bias}.
\graphicspath{{/Users/Diaraf/Desktop/M2_MIHP/GLCS/MiniProjectP/myProjects/Convolution/}}

\includegraphics(convolution.jpeg)
\includegraphics(exConvol.jpeg)


\section{Structure général du code et Etapes de conception}
Pour debuter, on prevois de commencer par une convolution simple pour le format {\it pgm}.
Avant de réaliser une extension du code  pour le format {\it ppm}.
N\^otre idée de d\'epart était de sutructure le code et ces étapes de conception  de la maniere suivante:

\begin{itemize}
\item Class Image
	\begin{itemize}
	\item Gestion du format PGM
	\item Gestion du format PPM
	\end{itemize}

\item Class Convolution
\item Fonction de kernel
\item Parallelisation
	\begin{itemize}
	\item MPI
	\item OpenMp
	\end{itemize}
\item Etablissement du systeme de benchmark.

\end{itemize}

IL est necessaire de rendre ce travaille le plus génerique que possible, pour pour voir s'amuser avec différent
type de convolution. Ainsi toute les class et fonction que nous implementerons dans la suite seront de types template.
\subsection{Remarque}
Dans la suite nous décrirons le de maniere plus d'\'etailler, les etapes de conceptions.

\section{Class Image}
Cette class sert au stocké en mémoire l'image sur laquelle on souhaite réaliser l'opération de convolution.
Pour cela on a du étudier le format Pixmap et on à essayer de  choisir  le systéme de stockage complet et le 
plus réetulisable par les autres {\it Class} du code.
La fonction principale de cette {\it Class} et celle du constructeur d'image . Elle permet d'enregistrer toutes 
les données de l'image dans les variables suivantes:
\begin{itemize}
	\item width   			//largeur
	\item height 			//longeur
	\item Maxval  			//Valeur Max 	
	\item mNumber 			//Numero Magique(P3 || P6)
	\item {\it std::vector}data 	// Tableua Valeur des pixels 
\end{itemize}
Pour que la convolution puisse r\'ecupérer ces paramettres il est aussi nécessaire d'implementer des {\it getter} pour 
chaque \'elements:
\begin{itemize}
\item getw()	    // getteur de width	
\item geth()	    // getteur de height	
\item getmv()	    // getteur de MaxVal	
\item getmn()	    // getteur de mNumber	
\item T operator() // getteur des valeur de pixel
\end{itemize}
Un setter sera aussi implemeneter pour stocké les resulats de l'operation de convolution dans une image cible.
\begin{itemize}
\item T \& operator() //setteur de pixel dans de la classe image.
\end{itemize}
\subsection{Redefinition d'operator}
En plus de la redéfinition de l'{\it op\'erator()}, les operateurs {\it <<}  et {\it >>}.
C'est derniers serviront respectivement en
\begin{itemize}
\item Afficher un élement de {\it type std::vector}
\item Engistrer dans un element de type {\it type std::vector}
\end{itemize}  

On utilisera  la fonction {\it std::copy} de la lib standard pour boucler sur les elements du {\it vector} de
destination.

\section{Class Convolution}
Cette classe réalise l'operation principale sur l'image. Elle réalise ainsi le filtrage et l'écriture du fichier
de sortie.
Pour mener \`a bien ces opérations, il faut ajouter à la convolution les variables:
\begin{itemize}
\item cible   //image cible
\item nkernel //taille du kernel
\item factor  //factor
\item bias    //bias	
\end{itemize}

La encore le constructeur de la {\it class} est l'élement le plus important, car c'est ça le coeur du programme.

\subsection{constructeur de convolution}
En se servant de nid de 4 boucles cette fonction recalcul tous les pixels de l'image en leur appliquant le kernel de convolution.
La version paralléle de ce code correpond juste au traitement d'un sous domaine de l'espace totale, aulieu de l'espace totale
entier. A la fin du calcul, chaque proc envoi la partie sur la  quel il a travaillé au process 0.
  

\subsection{Save}
Cette fonction n'est executer que dans le process 0 et elle permet d'écrire le fichier de sortie de la convolution 
sous le format de depart fourni(pgm, ppm).


\section{Environnement}
Machine : processeur Intel Haswel Quad core 2Ghz\\
		  cache size : 6144 KB \\	
		  Version GCC : 4.8.4\\
		  		
Pour nos testes on choisira Lesimages lena.pgm et flower.ppm  \\
On  utilise la fonction {\it MPI\_Wtime}, qui renvoit le nombre de temps d'exection en secondes

\section{Mesure sequentielle}


\section{makefile}



\section{Conclusion}

\end{document}






